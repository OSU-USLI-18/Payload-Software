\documentclass[onecolumn, draftclsnofoot,10pt, compsoc]{IEEEtran}
\usepackage{graphicx}
\usepackage{url}
\usepackage{setspace}
%\usepackage{bibtex}
\usepackage{geometry}
\geometry{textheight=9.5in, textwidth=7in}

% 1. Fill in these details
\def \CapstoneTeamName{		USLI CS Payload Subteam}
\def \CapstoneTeamNumber{		33}
\def \GroupMemberOne{			Joseph Struth}
\def \GroupMemberTwo{			Mark Bereza}
\def \GroupMemberThree{			Kevin Turkington}
\def \CapstoneProjectName{		NASA University Student Launch Initiative}
\def \CapstoneSponsorCompany{	Mechanical Engineering, Oregon State University NASA}
\def \CapstoneSponsorPerson{		Dr. Nancy Squires}

% 2. Uncomment the appropriate line below so that the document type works
\def \DocType{		Problem Statement
				%Requirements Document
				%Technology Review
				%Design Document
				%Progress Report
				}
			
\newcommand{\NameSigPair}[1]{\par
\makebox[2.75in][r]{#1} \hfil 	\makebox[3.25in]{\makebox[2.25in]{\hrulefill} \hfill		\makebox[.75in]{\hrulefill}}
\par\vspace{-12pt} \textit{\tiny\noindent
\makebox[2.75in]{} \hfil		\makebox[3.25in]{\makebox[2.25in][r]{Signature} \hfill	\makebox[.75in][r]{Date}}}}
% 3. If the document is not to be signed, uncomment the RENEWcommand below
%\renewcommand{\NameSigPair}[1]{#1}

%%%%%%%%%%%%%%%%%%%%%%%%%%%%%%%%%%%%%%%
\begin{document}
\begin{titlepage}
    \pagenumbering{gobble}
    \begin{singlespace}
    	%%\includegraphics[height=4cm]{coe_v_spot1}
        \hfill 
        % 4. If you have a logo, use this includegraphics command to put it on the coversheet.
        %\includegraphics[height=4cm]{CompanyLogo}   
        \par\vspace{.2in}
        \centering
        \scshape{
            \huge CS Capstone \DocType \par
            {\large\today}\par
            \vspace{.5in}
            \textbf{\Huge\CapstoneProjectName}\par
            \vfill
            {\large Prepared for}\par
            \Huge \CapstoneSponsorCompany\par
            \vspace{5pt}
            {\Large\NameSigPair{\CapstoneSponsorPerson}\par}
            {\large Prepared by }\par
            Group\CapstoneTeamNumber\par
            % 5. comment out the line below this one if you do not wish to name your team
            \CapstoneTeamName\par 
            \vspace{5pt}
            {\Large
                \NameSigPair{\GroupMemberOne}\par
                \NameSigPair{\GroupMemberTwo}\par
                \NameSigPair{\GroupMemberThree}\par
            }
            \vspace{20pt}
        }
        \begin{abstract}
        % 6. Fill in your abstract    
        	This document is written using one sentence per line.
        	This allows you to have sensible diffs when you use \LaTeX with version control, as well as giving a quick visual test to see if sentences are too short/long.
        	If you have questions, ``The Not So Short Guide to LaTeX'' is a great resource (\url{https://tobi.oetiker.ch/lshort/lshort.pdf})
        \end{abstract}     
    \end{singlespace}
\end{titlepage}
\newpage
\pagenumbering{arabic}
\tableofcontents
% 7. uncomment this (if applicable). Consider adding a page break.
%\listoffigures
%\listoftables
\clearpage

% 8. now you write!
\section{Definition and Problem Description}
The main goal of NASA's University Student Launch Initiative (USLI) competition is to launch a rocket exactly one mile into the atmosphere with a "experiment" on board of one of the following: "target destination", autonomous "deployable rover", or "landing coordinates via triangulation" CITE NASA. Oregon State University experiment of choice is the deployable rover. The rover experiment only has two minimum requirements set by NASA. Upon landing and locating the rocket the team will remotely trigger a deployment procedure of the rover. Once deployed the rover must autonomously (without any human assistance) move from its deployment zone to its destination at least five feet away from the rocket. Upon reaching its destination or before battery depletion the rover will then deploy solar panels. As a side goal of the USLI competition teams must conduct educational outreach both direct or indirect with schools or clubs. Each member of the USLI team will participate in at least one outing of educational outreach (as defined by NASA in the USLI handbook) at a local or non local school(K - 12).

\begin{quote}
Payload Requirements from USLI handbook:\\
4.5.1) Teams will design a custom rover that will deploy from the internal structure of the launch vehicle.\\
4.5.2) At landing, the team will remotely activate a trigger to deploy the rover from the rocket.\\
4.5.3) After deployment, the rover will autonomously move at least 5 ft. (in any direction) from the launch vehicle.\\
4.5.4) Once the rover has reached its final destination, it will deploy a set of foldable solar cell panels.\\
\end{quote}\cite{USLI_handbook}

\begin{quote}
Educational Outreach Definitions from USLI Handbook:\\
Direct interactions: A count of participants in instructional, hands-on activities where participants engage in learning a STEM topic by actively participating in an activity. This includes instructor- led facilitation around an activity regardless of media (e.g. DLN, face-to-face, downlink.etc.). Example: Students learn about Newton’s Laws through building and flying a rocket.\\ \\
Indirect interactions: A count of participants that interact with the team. For example: The team sets up a display at the local museum during Science Night. Students come by and talk to the team about their project.\\
\end{quote}\cite{USLI_handbook}
The minimum metrics set by NASA in their USLI handbook for the rover (5 ft of movement and deployable solar panels) are specific to the Computer science (software) and Electrical Engineering (hardware) members of the payload sub team. Other metrics such as deployment of the rover, creation of the rocket containing the payload, avionics and recovery of OSU's USLI rocket are specific too their respective sub teams. 

\section{Proposed Solution}
\subsection{Rover}
For our proposed solution of the Oregon State University USLI rover the payload sub team will design a two wheeled rover with the ability to fit within the rocket designed by the structural team. The rover will be fitted with bumper sensors as well as either Lidar or Radar sensors to avoid collision with objects within its travel path. The rover itself will be controlled by either a Raspberry Pi Zero or Raspberry Pi B depending on the payload team electrical engineers choice. The micro controller itself will be imaged with ROS (the Robot Operating System) for the ability to conduct SLAM (Simultaneous Localization and Mapping) operations. To assist with the rovers maneuvering algorithm as well as creating a mapping of its immediate surroundings.  
\subsection{Educational Outreach}
Collectively the Oregon State University USLI team will conduct at least one session of educational outreach at a local school per month. Demonstrating varying general scientific topics which may or may not include topics pertaining to rocketry. With the goal of conducting science experiments or presentations with a minimum of 200 children ranging from kindergarten to high school. 
\section{Performance Metrics}
\subsection{General}
The bare minimum requirements for the computer science payload sub team is to program a rover with the capability of moving at least 5 ft autonomously after activation from a remote trigger developed by the deployment sub team. Upon reaching its final destination or before battery depletion the rover will then deploy its solar panels for its final stage. Because the rocket launch is dependent on the success of launch, recovery, and deployment, these metrics will be measured during the final (non rocket launch) test of the rover. In addition to conducting in at least one day of educational outreach (as defined in the USLI handbook) at a local school (K-12).
\subsection{Optional}
The computer science payload sub team, optional metrics provided by Dr. Squires and other sub teams within the Oregon State University USLI team. Will maintain the OSU USLI competition website. In addition to providing an interface for in flight rocket data acceptable by the avionics sub team. As well as programming the rover with the ability to drive up to the USLI judges during the Huntsville Alabama competition.
% bibliography
\nocite{*}%if nothing is referenced it will still show up in refs
\bibliographystyle{plain}
\bibliography{bibliography}
%end bibliography

\end{document}



