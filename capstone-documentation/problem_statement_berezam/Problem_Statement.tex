\documentclass[onecolumn, draftclsnofoot,10pt, compsoc]{IEEEtran}
\usepackage{graphicx}
\usepackage{url}
\usepackage{setspace}
\usepackage{indentfirst}
\usepackage{geometry}
\geometry{textheight=9.5in, textwidth=7in}

% 1. Fill in these details
\def \CapstoneTeamName{		Code Monkeys in Space}
\def \CapstoneTeamNumber{		33}
\def \GroupMemberOne{			Mark Bereza}
\def \GroupMemberTwo{			Joseph Struph}
\def \GroupMemberThree{			Kevin Turkington}
\def \CapstoneProjectName{		NASA University Student Launch Initiative}
\def \CapstoneSponsorCompany{	Mechanical Engineering, OSU}
\def \CapstoneSponsorPerson{		Dr. Nancy Squires}

% 2. Uncomment the appropriate line below so that the document type works
\def \DocType{		Problem Statement
				%Requirements Document
				%Technology Review
				%Design Document
				%Progress Report
				}
			
\newcommand{\NameSigPair}[1]{\par
\makebox[2.75in][r]{#1} \hfil 	\makebox[3.25in]{\makebox[2.25in]{\hrulefill} \hfill		\makebox[.75in]{\hrulefill}}
\par\vspace{-12pt} \textit{\tiny\noindent
\makebox[2.75in]{} \hfil		\makebox[3.25in]{\makebox[2.25in][r]{Signature} \hfill	\makebox[.75in][r]{Date}}}}
% 3. If the document is not to be signed, uncomment the RENEWcommand below
\renewcommand{\NameSigPair}[1]{#1}

%%%%%%%%%%%%%%%%%%%%%%%%%%%%%%%%%%%%%%%
\begin{document}
\begin{titlepage}
    \pagenumbering{gobble}
    \begin{singlespace}
        \hfill 
        % 4. If you have a logo, use this includegraphics command to put it on the coversheet.
        %\includegraphics[height=4cm]{CompanyLogo}   
        \par\vspace{.2in}
        \centering
        \scshape{
            \huge CS Capstone \DocType \par
            {\large\today}\par
            \vspace{.5in}
            \textbf{\Huge\CapstoneProjectName}\par
            \vfill
            {\large Prepared for}\par
            \Huge \CapstoneSponsorCompany\par
            \vspace{5pt}
            {\Large\NameSigPair{\CapstoneSponsorPerson}\par}
            {\large Prepared by }\par
            Group\CapstoneTeamNumber\par
            % 5. comment out the line below this one if you do not wish to name your team
            \CapstoneTeamName\par 
            \vspace{5pt}
            {\Large
                \NameSigPair{\GroupMemberOne}\par
                \NameSigPair{\GroupMemberTwo}\par
                \NameSigPair{\GroupMemberThree}\par
            }
            \vspace{20pt}
        }
        \begin{abstract}
        % 6. Fill in your abstract    
        	The purpose of NASA's USLI is to construct and launch a rocket that will go at least a mile above ground, safely land, and deploy a rover capable of autonomous movement that will deploy solar cells after moving at least 5 feet from the rocket. 
			In particular, the CS students on OSU's USLI team are responsible for designing, implementing, and testing all software necessary to accomplish this task. 
			This will include rover motor control and obstacle avoidance, graphical representation of rocket positional data during flight, and creation/maintainance of a website hosting project information and deliverables.
        \end{abstract}     
    \end{singlespace}
\end{titlepage}
\newpage
\pagenumbering{arabic}
\tableofcontents
% 7. uncomment this (if applicable). Consider adding a page break.
%\listoffigures
%\listoftables
\clearpage

% 8. now you write!
\section{Problem Definition}
The problem posed to the CS students taking part in OSU's USLI team is a small but important part of a much larger endeavour. 
Overall, the goal of the competition is to: 
\begin{enumerate}
\item Construct and launch a rocket carrying a payload at least a mile \(5,280 feet\) above ground
\item Have the rocket deploy a parachute and safely land within 2,500 feet of the launch point
\item After landing and after a button press, deploy a rover
\item Have the rover drive autonomously at least 5 feet away from the rocket landing site
\item Have the rover deploy functional solar cells after being at least 5 feet from the landing site
\item Create formal technical reports regarding rocket/payload design and present them to a NASA review panel during several formal design review meetings
\item Participate in educational outreach and get at least 200 individuals involved in the project
\item Maintain a website detailing project information and hosting all competition deliverables
\end{enumerate}
\par The subset of these overall goals that pertain to the CS capstone students involves the research, design, implementation, and testing of software for the rover, in-flight rocket positional information, and the aforementioned website. 
Additonally, the CS capstone students must contribute software design specifications and research to all reports and presentations required by the competition. 
These include:
\begin{enumerate}
\item Preliminary Design Review report \& presentation
\item Critical Design Review report \& presentation
\item Flight Readiness Review report \& presentation
\item Launch Readiness Review outline
\item Post-Launch Assessment Review outline
\end{enumerate}
\par Finally, the team as whole, CS students included, must engage at least 200 participants by the end of the competition through direct, instrucitonal, hands-on activities that serve to educate the individual in a STEM topic relating to the project.
\section{Proposed Solution}
Each proposed solution to the aforementioned problems can be seperated into one of four categories: website creation/maintainance, rover intelligence, rocket flight information, and outreach.
\subsection{Website Creation/Maintenence}
The website will be fairly minimalist in appearance and hosted on the custom domain osuusli.com. It will be contained on a single page and will be implemented with an attractive template. The site itself will contain, at a minimum:
\begin{itemize}
\item The team name and logo
\item A list of all participants and their roles in the project
\item A brief description of the project and our goals
\item Download/view links for all project deliverables in .PDF form
\item Download/view links for important design documents
\item A visual timeline of important events with dates, locations, and descriptions
\end{itemize}
\par Additionally, the website will be regularly updated as new information comes to light, membership grows, and deadlines come and go.
\subsection{Rover Intelligence}
In order to accomplish the broad goal of making the rover move autonomously, avoid obstacles, and deploy solar cells, the rover's microcontroller must be loaded with fairly sophisticated software.
\par In order to avoid the ever-present software development issue of reinventing the wheel, the Robot Operating System framework will be used to design software for the rover's intelligence.
The Robot Operating System provides extensive tutorials and libraries in \verb!C++!, Python, and Lisp specifically designed for robotics applications, making it an ideal tool for our purposes.
\par As for the programming language used for rover software, \verb!C++! will be used due to it's high performance, low memory overhead, its extensive standard libraries, and its support for multi-processing (a feature missing from most Python implementations). 
The microcontroller for the rover will be a Raspberry Pi, though the exact model has not yet been determined.
If a single-core Raspberry Pi is selected, Python will be strongly considered as an alternative to \verb!C++! due to its high readability, intuitive syntax, and its automatic memory management.
This is because its biggest weakness, its lack of multi-processing support, will no longer be a factor.
\par The operating system that will run on the Raspberry Pi will be Raspian, a Linux distribution specifically designed to run on a Raspberry Pi with support for ROS, making it the obvious choice for our purposes.
\par Software will be written for the rover that will allow it to read data from its radar sensors and construct occupancy grids.
These occupancy grids will be combined with a software-implemented movement algorithm to perform autonomous movement aided with obstacle avoidance.
Additonally, software will be designed to use sonar sensor data to determine the absolute distance and angle of the rocket relative to rover, which will allow it move away from the rocket and determine when it is at least 5 feet away.
Finally, software will be designed for the rover to eject the solar cells once it has determined it has sufficiently cleared the landing site.
\subsection{Rocket Flight Information}
The rocket will be equiped with an altimeter for elevation measurements and a GPS sensor. 
These sensors will communicate with a groud station so that the team may confirm the final height reached by the rocket and to track its location during flight to aid in locating the payload.
However, raw analog sensor data is not very useful to human eyes, so the data will undergo analog to digital conversion.
\par Software will then be written to take this digital flight data as input and use it to produce organized numerical data and two visual displays: one charting the rocket's height over time, and the other charting its two-dimensional GPS location. 
A mature object-oriented programming language is the proper fit for a GUI application such as this, so either Java or \verb!C#! will be used to development the GUI.
\subsection{Outreach}
The goal of reaching 200 people over the course of the competition will be reached through a combination of various strategies. 
At the moment, these include leveraging primary school contacts to schedule classroom demos and presentations, distributing fliers, and collobarating with OSU clubs like rocketry and robotics to spread the word about our amazing project.
Additionally, each member of the core team will volunteer to take an active role in educational outreach for two seperate months between now and the competition's conclusion.
\section{Performance Metrics}
The following performance metrics will help evaluate the success or failure of this undertaking:
\begin{itemize}
\item The website runs without erros and is publically accessible
\item Users of the website can view and download PDFs for all deliverables before the due date.
\item All members will be equiped with the necessary tools for teleconferencing, including microphone and camera
\item The rover will operate autonomously after the single button press for payload deployment
\item The rover will move at least 5 feet from the rocket's land site before deploying operational solar cells
\item The rover will drive its required distance without getting stuck on any obstacle for more than 15 minutes
\item A graphical representation of both the rocket's flight and its GPS position will be displayed to the team in psuedo-realtime during flight
\item All software written will be maintained using Git version control with remote repositories hosted on GitHub
\item All software changes to the master branch will only be made after review and approval from a team member
\item All code must compile without errors or warnings before deployment to the rover
\item The final software used in the rover will have at least \verb!90%! unit test code coverage
\item All software written for the rover will follow a single coding style
\item All members of the CS team will attend every NASA review meeting
\item The CS team members will conribute at least 20 pages of content to the PDR, CDR, and FRR reports describing the software utilized in the project, design decisions, and any pertinent research
\item All CS team members will abide by the rules set forth in the launch and safety checklist, the local rocketry club's RSO, the Federal Aviation Administration's laws, and the High Power Rocket Safety Code
\item The team as a whole will reach 200 people before the competition's conclusion
\end{itemize}
\end{document}