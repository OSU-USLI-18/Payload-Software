\documentclass[onecolumn, draftclsnofoot,10pt, compsoc]{IEEEtran}
\usepackage{url}
\usepackage{setspace}
\usepackage{hyperref}
\newenvironment{myitemize}
{ \begin{itemize}
    \setlength{\itemsep}{0pt}
    \setlength{\parskip}{0pt}
    \setlength{\parsep}{0pt}     }
{ \end{itemize}                  }
\usepackage{geometry}
\geometry{textheight=9.5in, textwidth=7in}
\def\code#1{\texttt{#1}}
% 1. Fill in these details
\def \CapstoneTeamName{Code Monkeys in Space}
\def \CapstoneTeamNumber{		33}
\def \GroupMemberOne{			Mark Bereza}
\def \GroupMemberTwo{			Joseph Struth}
\def \GroupMemberThree{			Kevin Turkington}
\def \CapstoneProjectName{		NASA University Student Launch Initiative}
\def \CapstoneSponsorCompany{	Mechanical Engineering, Oregon State University NASA}
\def \CapstoneSponsorPerson{		Dr. Nancy Squires}

% 2. Uncomment the appropriate line below so that the document type works
\def \DocType{	%	Technology Review and Implementation Plan
				%Requirements Document
				%Technology Review
				%Design Document
				Progress Report
				}
			
\newcommand{\NameSigPair}[1]{\par
\makebox[2.75in][r]{#1} \hfil 	\makebox[3.25in]{\makebox[2.25in]{\hrulefill} \hfill		\makebox[.75in]{\hrulefill}}
\par\vspace{-12pt} \textit{\tiny\noindent
\makebox[2.75in]{} \hfil		\makebox[3.25in]{\makebox[2.25in][r]{Signature} \hfill	\makebox[.75in][r]{Date}}}}
% 3. If the document is not to be signed, uncomment the RENEWcommand below
\renewcommand{\NameSigPair}[1]{#1}

%%%%%%%%%%%%%%%%%%%%%%%%%%%%%%%%%%%%%%%
\begin{document}
\begin{titlepage}
    \pagenumbering{gobble}
    \begin{singlespace}
    	%%\includegraphics[height=4cm]{coe_v_spot1}
        \hfill 
        % 4. If you have a logo, use this includegraphics command to put it on the coversheet.
        %\includegraphics[height=4cm]{CompanyLogo}   
        \par\vspace{.2in}
        \centering
        \scshape{
            \huge CS Capstone \DocType \par
            {\large\today}\par
            \vspace{.5in}
            \textbf{\Huge\CapstoneProjectName}\par
            \vfill
            {\large Prepared for}\par
            \Huge \CapstoneSponsorCompany\par
            \vspace{5pt}
            {\Large\NameSigPair{\CapstoneSponsorPerson}\par}
            {\large Prepared by }\par
            Group\CapstoneTeamNumber\par
            % 5. comment out the line below this one if you do not wish to name your team
            \CapstoneTeamName\par 
            \vspace{5pt}
            {\Large
                \NameSigPair{\GroupMemberOne}\par
                \NameSigPair{\GroupMemberTwo}\par
                \NameSigPair{\GroupMemberThree}\par
            }
            \vspace{20pt}
        }
        \begin{abstract}
        % 6. Fill in your abstract    
        	This document serves as a retrospective of all work done to the website, avionics DLM, and rover over the fall term.
        \end{abstract}     
    \end{singlespace}
\end{titlepage}
\newpage
\pagenumbering{arabic}
\tableofcontents
% 7. uncomment this (if applicable). Consider adding a page break.
%\listoffigures
%\listoftables
\clearpage

% 8. now you write!
\section{Purposes and Goals}
Overall, the goals of this competition are the following:
\begin{enumerate}
\item Construct and launch a rocket carrying a payload at least a mile (5,280 feet) above ground.
\item Have the rocket deploy a parachute and safely land within 2,500 feet of the launch point.
\item After landing and after a button press, deploy a rover.
\item Have the rover drive autonomously at least 5 feet away from the rocket landing site.
\item Have the rover deploy solar cells after being at least 5 feet from the landing site.
\item The solar cells will increase in surface area after deployment (unfold).
\item Create formal technical reports regarding rocket/payload design and present them to a NASA review panel
during several formal design review meetings.
\item Participate in educational outreach and get at least 200 individuals involved in the project.
\item Maintain a website detailing project information and hosting all competition deliverables.
The subset of these overall goals that pertain to the CS capstone students involves the research, design, implementation.
\end{enumerate}
In particular, the CS seniors on the OSU USLI team, otherwise known as Code Monkeys in Space, are responsible for goal 9, their shares of goals 7 and 8, and the software needed to facilitate goals 4 and 5. Furthermore, Code Monkeys in Space are responsible for the design and implementation of software that will run on a data logging module inside every test launch vehicle for the purposes of data collection, which indirectly assist with goals 1 and 2. Thus, the scope of this project for Code Monkeys in Space is limited to these facets of the competition. The others will be handled by the mechanical and electrical engineering seniors on the team. If every subteam successfully completes their responsbilities in a timely, complete, and safe manner, the team as a whole will accomplish the overall goal of scoring well in this competition.
\section{Progress Thus Far}
\subsection{General}
Preliminary Design Review score sheets are in; OSU USLI team finished in the top 20\% in the competition which equates to 11th out of 50 teams! This result is very good for a first year rookie team. CS Team members contributed to the 200 page document by writing the verification plan, creating the website to post deliverables, and writing the software design section.
\subsection{Educational Outreach}
The team as a whole conducted three educational outreach events, with Code Monkeys in Space participating in two of these three events. The team as a whole has already exceeded the education outreach minimum requirement of 200 students with plenty of time to go before the competition's conclusion.
\begin{itemize}
\item Silver Crest School (K8) - The team hosted over 35 eighth-graders here at OSU. Together, the team members and the middle school students built and launched model rockets (A-motors) on the OSU campus after a presentation about rocketry. Additionally, the team provided a tour of the local AIAA chapter lab space.
\item Philomath Middle School (K6-7) - A science classroom of over 30 sixth- and seventh-graders was given a presentation about rocketry and STEM at OSU. Additionally, the team conducted a Q\&A session and decorated and launched model rockets!
\item Sprague High School (K9-12) - To over 200 high schoolers, over the course of four class periods, the USLI team conducted electrical and magnetic science experiments. The team also gave a presentation about rocketry and STEM at OSU. Finally, model rockets were decorated and launched.
\end{itemize}
\subsection{Rover}
First, the team decided which microcontroller, operating system, software framework, and programming language were the best choices for software development for the rover researching several alternatives for each category. After deciding on a Raspberry Pi running Raspbian and ROS with code written in C++/Python, Raspbian was installed on a Raspberry Pi and connected to internet via a subdomain of the domain name used for the team website. Additionally, ROS was installed on the Pi and accounts were made for every team member planning to work on the rover software. Said team members also completed some Robot Operating System tutorials for practice. Finally, Code Monkeys in Space start creating some basic hardware-independent simulations of the rover's movement in ROS.
\subsection{Avionics}
The team researched and selected the Adafruit Python IO Library made for the BeagleBone Black micorcontller which supports GPIO, PWM, ADC, I2C, SPI, and UART serial communication. Additionally, the development plan was created, which starts with getting reading/storing data to work one sensor at a time and integrate more as they arrive and as funding allows. Overall, the Data Logging Module is on track to be ready for sub scale launch before the Critical Design Review.
\subsection{Website}
The osuusli.com domain name was purchased and the team website (hosted via GitHub.io on the team's repository) was made accessible to the public via this url. USLI team information and useful links wer also made available via a sidebar on the website. Final Website designs were also implemented to meet NASA and competition requirements. Most recently, the project proposal and the PDR document, presentation, and flysheet were posted on the team website and made available for both online viewing and download. Beyond that, slight aesthetic improvements were made to the website along the way, including reorganizing the textbox layout for team member emails and making the team name appear more clearly against the various background images.
\section{Stumbling Blocks and Solutions}
\subsection{General}
\begin{itemize}
\item Problem: From our Preliminary Design Review (PDR), we received only 40\% of the total possible points for the safety category.
\begin{itemize}
\item Solution: We increased focus on safety by all team members, and began to rewrite environmental safety bullet points in document (as we had previously misunderstood the expectations).
\end{itemize}
\item Problem: Underclassmen leaving or losing interest in the USLI competition as a result of this term being almost entirely composed of designing, documentation, and planning.
\begin{itemize}
\item Solution: Now that parts have arrived and more hands-on work is available, we can begin trying to reach out to underclassmen again and promise them more fun work to do than simply attending design meetings.
\end{itemize}
\end{itemize}
\subsection{Rover}

\begin{itemize}
\item Problem: Robotics Operating System (ROS) learning curve is greater than anticipated.
\begin{itemize}
\item Solution: Work on simulations, using other team members' ROS simulation code from their robotics class as a starting point.
\end{itemize}
\item Problem: Serious development delayed until the rover testbed was completed.
\begin{itemize}
\item Solution: Testbed finished at the end of November. Development proceeded using hardware-independent ROS simulations until then.
\end{itemize}
\end{itemize}
\subsection{Avionics}
\begin{itemize}
\item Problem: Lack of initial funding prevented expensive data sensors from being ordered sooner.
\begin{itemize}
\item Solution: NASA funding grant approved on 11/28/17, parts were ordered soon after.
\end{itemize}
\item Problem: Sensors to be used in the Data Logging Module were unknown, which hindered development.
\begin{itemize}
\item Solution: Sensor decisions finalized by avionics subteam lead and Code Monkeys in Space on 11/03/17.
\end{itemize}
\item Problem: Development on the Data Logging Module bottlenecked by lack of hardware.
\begin{itemize}
\item Solution: Started research on serial communication libraries until sensors arrived.
\end{itemize}
\end{itemize}
\subsection{Website}
\begin{itemize}
\item Problem: Details for expected features and website design were unknown and changed frequently. 
\begin{itemize}
\item Solution: Discussed desired website layout with team leads during October and the design was finalized before the PDR was due in November.
\end{itemize}
\item Problem: Difficulties in making web development work available to underclassmen USLI team members.
\begin{itemize}
\item Solution: Github issues were created to help organize and delegate desired work.
\end{itemize}
\end{itemize}
\section{Weekly Worklog}
\begin{singlespacing}
\begin{tabular} {l p{0.45\linewidth} p{0.45\linewidth}} \textbf{Week} & \textbf{Work Done} & \textbf{Problems Encountered}\\\hline
1 &
\vspace{-\baselineskip}\begin{myitemize}
\item Created tutorial for Git workflow.
\item GitHub repositories setup for team.
\item Joined team Google Drive, Slack, and Trello.
\item Research Raspberry Pi performance/hardware limitations.
\vspace{-\baselineskip}\end{myitemize} & 
\vspace{-\baselineskip}\begin{myitemize}
\item Desired GitHub organization and permissions unclear.
\item Team existed for weeks before CS members joined, so CS members had to play catchup.
\vspace{-\baselineskip}\end{myitemize} \\\hline
2 &
\vspace{-\baselineskip}\begin{myitemize}
\item Raspberry Pi 3 selected as microcontroller.
\item Presented tutorial on Git workflow.
\item Started work on team website.
\item Signed up for outreach event.
\item Filled out Media Release Forms for NASA.
\item Purchased osuusli.com domain name.
\vspace{-\baselineskip}\end{myitemize} & 
\vspace{-\baselineskip}\begin{myitemize}
\item Don't yet have access to Raspberry Pi.
\vspace{-\baselineskip}\end{myitemize} \\\hline
3 &
\vspace{-\baselineskip}\begin{myitemize}
\item Finished Problem Statement rough draft..
\item Attended NASA kickoff teleconference.
\item Attended tour of Blue Origin.
\item Attended PDR review session w/ Dr. Squires and John Lindal.
\vspace{-\baselineskip}\end{myitemize} & 
\vspace{-\baselineskip}\begin{myitemize}
\item Potentially antagonistic volunteer joined team.
\vspace{-\baselineskip}\end{myitemize} \\\hline
4 & 
\vspace{-\baselineskip}\begin{myitemize}
\item Reserved room for CS-specific team meetings.
\item Onboarding of underclassmen and exchange of contact info.
\item Setup ROS on Raspberry Pi, started tutorials.
\item Finished Problem Statement final draft.
\item Created high level state diagram for rover movement.
\item Began discussion of data logger requirements/design.
\vspace{-\baselineskip}\end{myitemize} & 
\vspace{-\baselineskip}\begin{myitemize}
\item Many of the underclassmen unfamiliar with software tools.
\item Scheduling a CS meeting time that works for everyone.
\vspace{-\baselineskip}\end{myitemize} \\\hline
5 &
\vspace{-\baselineskip}\begin{myitemize}
\item Defined what sensors will be used in the DLM.
\item Prebuilt rocket parts for educational outreach activity.
\item Educational outreach activity w/ Silver Crest students.
\item Wrote software section of PDR document.
\vspace{-\baselineskip}\end{myitemize} & 
\vspace{-\baselineskip}\begin{myitemize}
\item Hard to write code for rover/DLM without hardware prototypes.
\vspace{-\baselineskip}\end{myitemize} \\\hline
6 &
\vspace{-\baselineskip}\begin{myitemize}
\item Finished writing/editing PDR document.
\item Finished Requirements Document.
\item Posted competition deliverables on website.
\vspace{-\baselineskip}\end{myitemize} & 
\vspace{-\baselineskip}\begin{myitemize}
\item Delay on shipping of rover/DLM parts.
\item Many parts not ordered due to lack of funds.
\vspace{-\baselineskip}\end{myitemize} \\\hline
7 &
\vspace{-\baselineskip}\begin{myitemize}
\item Finished high level design of rover movement algorithm.
\item Finished technology review rough drafts.
\item Contacted Sprague High School about educational outreach.
\vspace{-\baselineskip}\end{myitemize} & 
\vspace{-\baselineskip}\begin{myitemize}
\item None.
\vspace{-\baselineskip}\end{myitemize} \\\hline
8 &
\vspace{-\baselineskip}\begin{myitemize}
\item Gave PDR presentation to NASA judges.
\item Completed technology review final drafts.
\item Attended ROS tutorial session.
\item Researched sensors and Python IO libraries for DLM.
\vspace{-\baselineskip}\end{myitemize} & 
\vspace{-\baselineskip}\begin{myitemize}
\item Low turnout of CS underclassmen. 
\vspace{-\baselineskip}\end{myitemize} \\\hline
9 &
\vspace{-\baselineskip}\begin{myitemize}
\item Completed Design Document rough draft.
\item Educational outreach w/ Philomath Middle School.
\item Started work on ROS simulations for rover movement.
\vspace{-\baselineskip}\end{myitemize} & 
\vspace{-\baselineskip}\begin{myitemize}
\item Mark gone this week.
\vspace{-\baselineskip}\end{myitemize} \\\hline
10 &
\vspace{-\baselineskip}\begin{myitemize}
\item Finished Design Document final draft.
\item Created progress report slides.
\item Recorded voiceover for progress report slides.
\vspace{-\baselineskip}\end{myitemize} & 
\vspace{-\baselineskip}\begin{myitemize}
\item Initially recorded video in front of projection of slides instead of screen capture.
\vspace{-\baselineskip}\end{myitemize} \\\hline
\end{tabular}
\end{singlespacing}
\section{Retrospective}
\begin{singlespacing}
\begin{tabular} {p{0.3\linewidth} p{0.3\linewidth} p{0.3\linewidth}} \textbf{Positives} & \textbf{Deltas} & \textbf{Actions}\\\hline
\vspace{-\baselineskip}\begin{myitemize}
\item Successfully qualified for the comptetion.
\item Scored in the top 20\% of teams for the PDR despite being a rookie team.
\item Already completed website for the competition on our custom domain name.
\item Educational outreach requirements for competition already met, with many more activities planned for future months.
\item At least two dedicated CS underclassmen will be assisting us with software going forward.
\item Won over \$10,000 for the project via Space Grant.
\item Got to attend tour of Blue Origin.
\item All capstone writing assignments completed on time.
\vspace{-\baselineskip}\end{myitemize} & 
\vspace{-\baselineskip}\begin{myitemize}
\item Safety scoring needs to improve for CDR and FRR.
\item Need method to test rover code directly and regularly on target hardware.
\item Get more underclassmen involved with the project.
\item Team needs social media presence.
\item Rover needs to move based on sensor input.
\item DLM needs to record in-flight data from sensors during test flights.
\item CS team members need to help more with educational outreach.
\vspace{-\baselineskip}\end{myitemize} & 
\vspace{-\baselineskip}\begin{myitemize}
\item Consultation with the safety officer will be done before working with any potentially dangerous hardware.
\item Test playpen will be created for rover with webcam to allow for remote development and testing.
\item Once more hands-on work is available, team will advertise competiton participation to peers.
\item Social media integration (Twitter or Facebook) will be added to the team website.
\item Movement and sensors communication modules will be created in ROS.
\item Python code will be written for DLM that will write sensor data during flight.
\item Code Monkey in Space will particiapte in at least one more educational outreach activity.
\vspace{-\baselineskip}\end{myitemize}
\end{tabular}
\end{singlespacing}
\end{document}
